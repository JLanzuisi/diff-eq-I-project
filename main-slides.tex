\documentclass{scrartcl}

\errorstopmode

\usepackage[spanish]{babel}
\usepackage{etoolbox}
\usepackage{ifthen}
\usepackage
	[
		usenames,
		dvipsnames,
	]
{xcolor}
\usepackage
	[
		colorlinks=true,
		allcolors=blue,
	]
	{hyperref}
\usepackage{url}
\usepackage{mathtools}

\usepackage{unicode-math}
\setmainfont{Latin Modern Roman}
\setsansfont
	[ Scale=MatchLowercase, ]
	{Josefin Sans Light}

\KOMAoptions
	{
		paper=a4,
		paper=landscape,
		fontsize=20,
		DIV=calc,
		abstract=yes,
		bibliography=openstyle,
		toc=flat,
		parskip=half,
		titlepage=yes,
	}

\setkomafont{title}{\normalfont\sffamily}
\setkomafont{subtitle}{\normalfont}
\setkomafont{subject}{\normalfont\normalsize\addfontfeatures{LetterSpace=10}}
\setkomafont{date}{\normalfont\normalsize}
\setkomafont{section}{\normalfont\sffamily\Large}

\RedeclareSectionCommands
  [
    tocentryformat=\normalfont,
    tocpagenumberformat=\normalfont
  ]
  {section}

\author{Jhonny Lanzuisi}
\title{\LARGE Cadenas de Markov de tiempo continuo}
\subtitle{Ecuaciones diferenciales en procesos estocásticos}
\subject{ECUACIONES DIFERENCIALES I}
\date{\today}
\titlehead{Universidad Simón Bolívar\hfill Matemáticas Puras y Aplicadas}

\newenvironment{definicion}
	{
		\textbf{Definición}.
	}
	{}

\newcounter{teo}
\newenvironment{teorema}
	{
		\textbf{Teorema}.
	}
	{}

\newenvironment{demostracion}
	{
		\emph{Demostración}.
	}
	{}

\newcommand{\Prob}
	{
		\mathbb{P}
	}

\newcommand{\Mat}[1][]
	{
		\ifthenelse{\equal{#1}{}}
			{\mathbf{P}}
			{\mathbf{#1}}
	}

\newcommand{\SepRule}
	{
		\par
		\rule{.35\linewidth}{.8pt}\hfill
	}

\makeatletter
	\newcommand{\pushright}[1]
		{
			\ifmeasuring@#1\else\omit\hfill$\displaystyle#1$\fi\ignorespaces
		}
	\newcommand{\pushleft}[1]
		{
			\ifmeasuring@#1\else\omit$\displaystyle#1$\hfill\fi\ignorespaces
		}
\makeatother

\pagestyle{empty}
\raggedbottom

\begin{document}
\maketitle

\section{Introducción}
Muchos procesos aleatorios operan en espacios discretos
pero cambian su valor en cualquier instante de tiempo,
en vez de intervalos regulares: átomos radioactivos que decaen,
la cantidad de moléculas en una reacción química,
poblaciones con nacimientos/muertes, etc.

Estos procesos lucen como una función a trozos con saltos
que ocurren en unidades de tiempo continuas.
Estos procesos pueden modelarse elegantemente usando
las cadenas de Markov de tiempo continuo.

Esta presentación dará las nociones básicas de dichas
cadenas de markov hasta llegar a la ecuación de
Kolmogorov, donde se ve la relación entre estos
procesos estocásticos y las E.\,D.\,O. de primer orden.

\section{Definición básica}

\begin{definicion}
	Familia de variables aleatorias que en espacio finito (numerable)
	que satisfacen la propiedad de Markov:
	\begin{align*}
		& \Prob\bigl(X_{t_n} = i_n\mid X_{t_1}=i_1,\dots,X_{t_{n-1}}=i_{n-1}\bigr)\\
		=\;&
		\Prob\bigl(X_{t_n} =i_n\mid X_{t_{n-1}}=i_{n-1} \bigr)
	\end{align*}
	para todo $i_1,\dots,i_n\in S$ y cualquier suseción $0\leq t_1\leq\dots\leq t_n$
	de instantes de tiempo.

	El proceso es \emph{homogeneo respecto del tiempo} si la probabilidad condicional
	no depende del instante actual, esto es:
	\begin{align*}
		 & \Prob\bigl(X_{t+s} = j\mid X_s = i\bigr) \\
		=&\; \Prob\bigl(X_t = j\mid X_0 = i\bigr),\quad s\geq0.
	\end{align*}
\end{definicion}

\pagebreak
\section{Probabilidades de transición}

\begin{definicion}
	La probabilidad de transición de una cadena homogena es:
	\[
		P_{ij}(t) = \Prob\bigl( X_{t+s} = j\mid X_s = i\bigr).
	\]
	La matriz cuyas componentes $(i,j)$ son $P_{ij}(t)$ es la
	\emph{matriz de transición}.
\end{definicion}

Ecuación de Chapman-Kolmogorov:

\begin{teorema}
	Sea $\Mat(t)$ la matriz de transición de una cadena homogenea,
	entonces:
	\[
		\Mat(t+s) = \Mat(t)\Mat(s),
	\]
	si, y solo si,
	\[
		P_{ij}(t+s) = \sum_{k\in S} P_{ik}(t) P_{kj}(s)
	\]
\end{teorema}

\section{Generador infinitesimal}

\begin{definicion}
	El generador de la cadena de Markov es la matriz:
	\[
		\Mat[Q] = \lim_{h\to0^+} \frac{\Mat(t) - \Mat[I]}{h},
	\]
	donde $\Mat[I]$ es la matriz identidad (recordemos que $\Mat(0)=\Mat[I]$).
\end{definicion}

\begin{teorema}
	Sea $q_{ij}$ la entrada $(i,j)$ de la matriz generadora.
	\begin{enumerate}
		\item La suma de los elementos en cada fila es 0,
			esto es, $\sum_j q_{ij} = 0$.
		\item $q_{ij}\geq0$ para $i\neq j$.
		\item $q_{ii}<0$.
	\end{enumerate}
\end{teorema}

\section{Ecuación de Kolmogorov}

\begin{teorema}
	Dada una cadena de Markov de tiempo continuo con generador
	$\Mat[Q]$ la matriz de transición $\Mat(t)$ cambia de acuerdo
	con la siguiente expresión:
	\[
		\Mat'(t) = \Mat(t)\Mat[Q].
	\]
\end{teorema}

Se puede resolver de forma explícita:
\[
	\Mat(t) = e^{\Mat[Q]t},
\]
\pagebreak
\appendix
\setbibpreamble
	{%
		La mayoría de las referencias provienen de
		transcripciones de cursos de distintas universidades.
		Todos los materiales usados se pueden conseguir en formato
		PDF con el enlace que se especifica.%
	}

\begin{thebibliography}{}
	\bibitem{holmes}
		Miranda Holmes-Cerfon.
		``Lecture 4: Continuous-time Markov Chains".
		2019.
		\url{https://cims.nyu.edu/~holmes/teaching/asa19/handout_Lecture4_2019.pdf}
	\bibitem{lalley}
		Steven P. Lalley.
		``Continuos Time Markov Chains".
		2013.
		\url{https://galton.uchicago.edu/~lalley/Courses/313/}
	\bibitem{takahara}
		Glen Takahara.
		``Continuous-Time Markov Chains - Introduction"
		2017.
		\url{https://mast.queensu.ca/~stat455/lecturenotes/set5.pdf}
		
\end{thebibliography}

\end{document}

